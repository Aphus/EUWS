\begin{abstract}

    The North Atlantic Oscillation (NAO) is an important driver of windstorm strength during the winter season. In this study we investigate the statistical relationship between the NAO and European windstorms. All data used to generate results are derived from the ERA5 and ERA5-Land reanalysis. We find that the frequency of windstorms is dominated by seasonal factors, while their intensity is determined by both seasonal factors and the NAO, particularly during the winter period. Windstorms associated with a positive NAO state are the most energetic, followed by those with a neutral NAO state. Negative NAO windstorms were consistently the least energetic. The length of a dataset significantly affects both the observed intensity of windstorms and their distribution in relation to the NAO. Datasets during a period with a positive NAO tendency will observe higher energy for windstorms associated with all NAO types, while the opposite holds for datasets during periods of negative NAO tendency. The NAO is found to not affect the frequency of windstorms, but a dataset will observe the majority of windstorms occurring during the the same NAO phase as the one the period over which the dataset is conducted tends to. The paths of positive NAO storms follow a meridional trajectory, while the path of negative NAO storms follows a zonal one. Neutral NAO storms follow a trajectory between meridional and zonal. 
    In this study, we develop a new storm severity index based on the energy of winds near the surface. This SSI is particularly useful for estimating the accumulation of damage during low-intensity windstorms with a long lifetime.

\end{abstract}
%\clearpage - not needed in documentclass report