\section{Discussion}

    \subsection{Risk of European Windstorm during NAO Neutral Index}

        In Figures \ref{fig:storm_count_vs_daily_nao} and \ref{fig:storm_count_vs_monthly_nao} we presented the number of windstorms associated with different states of the NAO. Our results support the findings of \cite{https://doi.org/10.1002/joc.1982} and \cite{https://doi.org/10.1002/2014GL059647}. In both cases, using daily or monthly NAO values, the maximum number of storms was observed in the range where the NAO index was between 0.0 and 0.5, although when using monthly indices the distribution is less pronounced than its daily index counterpart. This is due to the natural smoothing that occurs when using a value that encapsulates a longer timeframe. Daily values are more nuanced but can be noisy, whereas monthly values are smoother but also more reliable. As described in Chapter 3, the way we determine the NAO state associated with a windstorm is by taking the one at the start of an event, defined as 36 hours before the point of highest vorticity moves over land anywhere in western Europe, including the UK and excluding Iceland. Because 36 hours is an arbitrary period that was found to best work with the algorithm to define windstorm events, there is a degree of uncertainty as to how accurately the state of the NAO is defined. By repeating statistical analysis using monthly values, we minimise this uncertainty as the likelyhood that an event had its starting time out by 15 days (the average length of time an event would be from a different month) is insignifficant.

        Next, we determine the distribution of the NAO states themselves, given in Figures \ref{fig:nao_and_event_rate}A and \ref{fig:nao_and_event_rate}C, to investigate the relationship between daily / monthly NAO and the frequency of windstorms. The figures show a spread of NAO states nearly identical in shape to the spread of events in Figures \ref{fig:storm_count_vs_daily_nao} and \ref{fig:storm_count_vs_monthly_nao}. The distribution is not dissimilar to that produced by an oscillator between -3.5 and 3.5 with a small positive bias. If we obtain the ratio of windstorm events associated with a particular NAO state to the number of days with that NAO state, we obtain an even spread, as presented in Figures \ref{fig:nao_and_event_rate}B and \ref{fig:nao_and_event_rate}D. The daily rates vary between 5 and 10\%, while the monthly rates are smoother and vary between 4 and 6\%. At the tail of the distribution, for events associated with an NAO index of magnitude larger than 3, we observe a high probability that days with that NAO state will observe a windstorm - 20\% in Figure \ref{fig:nao_and_event_rate}B, however, the uncertainty is high in that range, as the number of events ($<5$), and the number of days of that state ($<30$) is very low compared to the duration of the study (70 years). We conclude that the state of the NAO, when estimated on the daily or monthly, has not displayed a significant relation to the frequency of windstorms in western Europe.

        A different angle and a similar conclusion is obtained when we consider the overlaying state of the NAO. In Section 5.2 we discuss the influence of the period and length of a dataset on the statistical relationship between windstorms and the NAO. Therein are Figure \ref{fig:yearlyNAO} and Table \ref{tab:eptable}. In the figure we present the yearly NAO, from 1950 to 2020, as the average of the monthly NAO. We are able to identify long periods of time when the NAO has a tendency towards the positive or negative, such as the range from 1950 to 1970 when the NAO was biased towards the negative and the range from 1970 to 2000 when the NAO was biased towards a positive state. The predisposition is not large in magnitude and is often $<5$, which we would normally classify as NAO neutral in the case of daily and monthly values; however, since these are the average of many values over a long period of time, we expect a more subtle variance and as such we focus on the sign and place less importance over the magnitude. 
        In Table \ref{tab:eptable} we present information relevant to the exceedance probability curves generated for the different periods we can identify in Figure \ref{fig:yearlyNAO}. We also list the number of events, split by their associated NAO state, that go into each period. In the previous paragraphs we discussed how the monthly NAO state has no relation to the frequency of windstorms, however, in Table \ref{tab:eptable} we record a significantly larger number of NAO(+) windstorms per year (7.5) during the positively biased period of 1970 to 2000 in comparison to the number of NAO(+) windstorms per year (4.9) during the negatively biased period of 1950 to 1970. 1970 to 2000 also observes a slightly higher number of NAO(0) events per year (6.9), compared to 1950 to 1970 (6.3). The opposite is true for NAO(-) events with the period of 1970-2000 observing only 5.1 windstorms/year of that type versus 7.5 windstorms/year during 1950-1970. Further research is needed to determine exactly what effect an unbiased period would have on windstorm frequency, as the above findings cannot distinguish a long enough period where the yearly average is close to 0. Note that the period from 2000 to 2020 is not truly neutral; it merely contains an even mixture of positive and negative years.

        The spread of events for the two periods discussed also differs from the spread of events in Figure \ref{fig:storm_count_vs_daily_nao} (we use daily NAO values to produce EP curves and therefore the table). Illustrated in the figure, NAO(+) events make up 34\% of events, NAO(0) - 44\%, and NAO(-) make up 22\%: NAO(+/0/-)=34/44/22. In the negatively biased period of 1950-1970 this ratio is NAO(+/0/-)=26/34/40 and in the positively biased 1970-2000 the ratio is NAO(+/0/-)=38/35/26. This shows that the distribution of event frequency has its maximum in the same NAO state as the bias of the period in which it resides. This is exactly what we would expect to see in a world where windstorm events were not related to the state of the NAO. An oscillator with its maxima in a particular state would most likely exist in that state during the occurrence of a separate event, with increasingly diminishing probability towards the minima. In this study, the oscillator is the NAO with its maxima determined by the period over which it is observed and constrained by the minima -3.5 and 3.5.

        In Figure \ref{fig:normalized_probability_of_storm_events_with_NAO} we observe a much more significant connection relating the frequency and seasonality of the windstorms. Windstorms in western Europe are most likely during the winter months of November to February and least likely in the summer months from May to August. This agrees with the consensus of previous studies in which the December-January period is established as the storm season \cite{Hurrell2003}, \cite{Pinto2009}, \cite{Palin2016}, \cite{Degenhardt2023}. 
        The distribution of events by NAO state for each month is similar to the distribution given in Figure \ref{fig:storm_count_vs_daily_nao}, which agrees with our conclusions so far. 

        So far we have concluded that through our analysis we have produced no statistical connection between the state of the NAO and the frequency of windstorm events. Strong corellation is observed between windstorm frequency and seasonality. Next, we will investigate the intensity of windstorms and their relation to the NAO state. In Figure \ref{fig:averageEnergyPerMonth} we present the average energy of windstorms for each month, split into their corresponding NAO states. A strong relation of windstorm severity is observed with both seasonality and the state of the NAO. The most energetic windstorms occur during the winter months. During those months, the most violent of storms occurred during a positive NAO, followed by neutral NAO events and with negative NAO events being the least energetic. During the summer months, this relationship between windstorm energy and the state of the NAO is not prominent, however, the number of events in each category is largely $<10$. In addition, if we consider the low energies of storms during that period, it becomes difficult to arrive at a meaningful conclusion.
        High-energy winter windstorm events during a positive NAO are also highlighted in Figure \ref{fig:EP_Curve_NAO(p,n)_1950-2020}, which shows the exceedance probability of windstorms from 1950 to 2020. As mentioned above, NAO(+) windstorms are most energetic, followed by NAO(0) and NAO(-). We conclude that NAO is an important driver of windstorm strength during the winter season. Seasonal effects also play an important role in the frequency and severity of windstorms.
    
    \subsection{The Influence of a Longer Dataset on Results}

        In the preceding section, we presented the annual NAO patterns as shown in Figure \ref{fig:yearlyNAO}. We identified three distinct periods: 1950-1970, characterised by a negative NAO bias; 1970-2000, marked by a positive NAO bias; and 2000-2020, featuring a balanced mix of both negative and positive states. We also introduced the exceedance probability (EP) curve, Figure \ref{fig:EP_Curve_NAO(p,n)_1950-2020}, covering the entire length of the study - 1950-2020. When discussing the probability that a windstorm event will take place during various states of the NAO, we also showed that the distribution of events associated with a particular NAO index is the result of the statistical distribution of the NAO states during the relevant period. And the statistical distribution is directly tied to the time frame over which it is taken, e.g. from 1970 to 2000 the NAO was predominantly positive, leading to the result in Table \ref{tab:eptable}, where the majority of windstorms during that time frame occurred while the NAO was in a positive state. This illustrates the importance of considering the period during which a study makes its statistical analysis. 

        We also established that NAO(+) events tend to be significantly more energetic than their counterparts, particularly during the winter season, which is when the majority of windstorms occur. Thus, we would expect a period with a predominantly positive NAO state to observe highly energetic windstorms. These expectations are confirmed in Figure \ref{fig:EP19702000}, which shows the highest average windstorm energy (Table \ref{tab:eptable}). The opposite is also true where the period of 1950 to 1970, biased towards a negative NAO state, observes the lowest average windstorm energy. Thus, the period over which a study is performed is important not only to the frequency of events but also for their severity.

        In Figure \ref{fig:EP20002020}, we see a very clear division between the curves associated with each of the NAO states. This is because the figure represents a mixture of years with a negative tendency where the majority of storms are during a NAO(-) phase and are therefore weak, and years with a positive tendency where the majority of storms are during a NAO(+), thus making them highly energetic. During each of the two periods, events of the opposite sign are the least likely to occur, which leads to a very clear weakly energetic NAO(-) EP curve and a clear NAO(+) EP curve. Of the curves, the one for NAO(0) events is the least affected by the choice of period. During a neutral NAO period, extratropical cyclones are least affected by the NAO and by proportion most affected by environmental factors which, when averaged on a decadal scale or above, are consistent and smooth when compared to the effects of a positive or negative NAO. 
    
    \subsection{The Influence of the NAO State on Windstorm Path Patterns in Europe}

        In this study, we focus on the damage potential from windstorms produced by an extratropical storm, rather than the accuracy of the path itself, which would be better tracked by identifying storms and tracking their point of highest vorticity. In this section, when we say the path of a storm, we specifically refer to the trail of highest windstorm damage associated with the storm, not the path of the storm itself. To analyse this path, we present three different points of view: a heat map of the accumulated energy from the windstorms during different NAO phases projected onto western Europe (Figures \ref{fig:naopostitivemap}, \ref{fig:naoneutralmapo} and \ref{fig:naonegativemapenter-label}), a statistical analysis that identifies which countries absorb the most energy from windstorms (Figures \ref{fig:DJF_Tot_Eng_by_cunt} and \ref{fig:JJA_Tot_Eng_by_cunt}), and a set of EP curves produced for all countries along the west coast of Europe (Figures \ref{fig:epcurvescountry}A-H).

        At the beginning of Chapter 5, we discussed the path of windstorms as seen in Figures \ref{fig:naopostitivemap}, \ref{fig:naoneutralmapo} and \ref{fig:naonegativemapenter-label}. The vast majority of storms pass through the UK, as supported by Figure \ref{fig:epcurvescountry}H, where the sum of the number of events - N adds up to 85\% of all storms in our dataset. Windstorms associated with a positive NAO phase continue through Denmark and into the Nordic countries drawing a strongly meridonal curve. Neutral NAO windstorms also move through Denmark, but their trajectory has a significantly weaker northward component as they head towards Lithuania, Latvia and Estonia. Negative NAO windstorms take a zonal path, passing through the Netherlands and through northern Germany in the direction of Poland and the Ukraine. 
        The path of extratropical storms will rarely match this precise description, the above are the accumulated energy from windstorms from a large dataset. However, these findings agree with \cite{Hurrell2003}, where positive NAO phases produce a meridonal trajectory for winds, while a negative NAO phase produces a zonal one.

        Figures \ref{fig:DJF_Tot_Eng_by_cunt} and \ref{fig:JJA_Tot_Eng_by_cunt}, we identify the countries which experience the most violent winds during the winter and summer season, respectively. For the winter season, those are Great Britain, Spain, Belgium and Norway, and for the summer season, they are Great Britain, Denmark, Norway and Belgium. Before we discuss this lists, and the presence of Spain in it in particular, we will introduce the last set of figures, as they will be required to obtain a better understanding of the results. These are Figures \ref{fig:epcurvescountry}A-H illustrating the EP curves for Spain, France, Belgium, the Netherlands, Denmark, Norway, Germany and Great Britain.

        Let us focus on the energy during the winter period, as that is the frame during which most storms occur. From the picture we have built so far, it is not surprising to see Great Britain, Belgium or Norway being on the list of countries that observe the highest percent of energy from windstorms. Particularly as they are the ones we named in the trajectory of NAO(+) storms which are, as discussed, the most energetic. If we look at Figure \ref{fig:epcurvescountry}A, we can observe the EP curve for Spain. In the figure are also listed the number of events that comprise each line. Spain experiences a nontrivial number of storms, but not enough to accumulate significantly at any one location. In Figure \ref{fig:naopostitivemap}, only several faint spots can be observed over Spain, which is due to the order of magnitude of the rest of the energies dealt with on the heatmap. The lower end is at $10^{15}$J, which is too high when distributing the energy over a long coastline.

        The opposite phenomenon is observed for Norway. In its EP curve in Figure \ref{fig:epcurvescountry}F we can observe that the winds energy is in the lower end, with less than 40\% of windstorms depositing more than $10^{15}$J and only 1 or 2 events with energy above $10^{16}$J. In Figure \ref{fig:DJF_Tot_Eng_by_cunt} it is disproportionately low compared to the brightness of the energy highlighted in the heatmap in Figure \ref{fig:naopostitivemap}. Although the energy share from individual events is low, Norway experiences winds above 15 m/s from nearly 80\% of the windstorms in the entire dataset. Unlike northern Spain, which has a relatively flat topographical profile, the mountainous terrain of Norway concentrates winds in select areas and allows for the accumulation of windstorm energy, which leads to its bright contour on the heatmap, yet relatively low energy in the statistical analysis. 

        The last country is notable for the lack of its presence in all three points of view. France, located further north than Spain, which we showed to hold a significant portion of windstorm energy, and with predominantly flat topography and a long coastline facing the Atlantic, the author anticipated a nontrivial build-up of energy from windstorms in this area. However, in Figure \ref{fig:epcurvescountry}B we observe that only 10\% of NAO(+) storms achieve more than $10^{15}$J despite the relatively large number of events (40\% of all storms brought on winds greater than 15 m/s in France). The author is unable to explain the anomalously low windstorm energy observed in France.
        
        
        
\section{Summary of Findings}

     The frequency of windstorms in western Europe is dominated by seasonal factors and no statistical significance is found relating frequency to the state of the North Atlantic Oscillation. The severity of windstorms is strongly related to the state of the NAO, particularly in the winter, when windstorms associated with a positive NAO state are distinctly energetic. Severity is also related to seasonal variables, which produce notably more energetic windstorms during winter than during summer. 

     The temporal range of a dataset is found to affect the statistical relationship of windstorms with the NAO, with the most windstorm events occurring during the state of the NAO that dominates during the range of that study. The dominating NAO state also affects the recorded energy of positive and negative NAO events, where studies made during a positively biased NAO recording higher energies and studies made during negatively biased NAO recording lower energies of windstorms.

     We find that more than 85\% of windstorm events that affect western Europe lead to winds exceeding 15 m/s in the UK. Extratropical storms associated with a positive NAO state follow a meridonal trajectory and the windstorms brought on by these storms are concentrated in, but not limited to, northern Europe. The topographical profile of Norway results in hotspots of windstorm energy accumulation, identified on a map in Figure \ref{fig:naopostitivemap}. Storms associated with a neutral NAO state follow a trajectory between a meridonal and zonal one, concentrating windstorm energy in the northern part of central Europe. Storms associated with a negative NAO state follow a zonal trajectory and concentrate windstorm energy over central Europe.

\section{Future Work}

    During this study, the author recognises the necessity of creating a database containing the dates, with start and end, of all extratropical storms that lead to the development of windstorms in Europe. This task requires tracking not only the centre of the storm, but its size as well, as some storms, such as the Great Storm of '87 run alongside the European coast in a meridonal path, but due to their size produce many destructive windstorms over land.

    In this study, we extrapolate the severity of windstorms during periods with a NAO biased towards the neutral state; however, a more reliable result might be produced by creating a decadal period like that through identifying and selecting individual years that are scattered over the entire dataset, as for the past 70 years the NAO has oscillated between a positive and negative state.

    The methods used in this experiment can be employed for further study by modifying the wind data used. An example would be to lower the threshold for wind speed from 15m/s for the ERA5 dataset to 13m/s, thus allowing a more thorough exploration into the accumulation of windstorm energy further into continental Europe. Alternatively, wind gust data would provide a more reliable measure than 1 hour wind speed, which can smooth out short but violent winds.
